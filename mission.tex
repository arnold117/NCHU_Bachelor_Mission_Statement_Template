%%
% NCHU Bachelor Proposal Report Template
%
% 南昌航空大学毕业设计任务书 —— 使用 XeLaTeX 编译
%
% Copyright 2023 Arnold Chow
%
% The Current Maintainer of this work is Arnold Chow.
%
% Compile with: xelatex -> biber -> xelatex -> xelatex

% 章节支持、单面打印:ctexbook
\documentclass[UTF8,AutoFakeBold,AutoFakeSlant,zihao=-4,oneside,openany]{ctexbook}
\usepackage[a4paper,left=2.5cm,right=2.5cm,top=2.5cm,bottom=2.5cm]{geometry}
% 目前 29mm 最接近 Word 排版
\usepackage{xeCJK}
\usepackage{titletoc}
\usepackage{fontspec}
\usepackage{setspace}
\usepackage{graphicx}
\usepackage{fancyhdr}
\usepackage{pdfpages}
\usepackage{setspace}
\usepackage{booktabs}
\usepackage{multirow}
\usepackage{caption}
\usepackage{tikz}
\usepackage{etoolbox}
\usepackage{hyperref}
\usepackage{xcolor}
\usepackage{caption}
\usepackage{array}
\usepackage{amsmath}
\usepackage{amssymb}
\usepackage{pdfpages}
\usepackage{float}
\usepackage[section]{placeins}
\usepackage{enumerate}
\usepackage{ulem}

% 设置参考文献编译后端为 biber,引用格式为 GB/T7714-2015 格式
% 参考文献使用宏包见 https://github.com/hushidong/biblatex-gb7714-2015
\usepackage[
  backend=biber,
  style=gb7714-2015,
  gbalign=gb7714-2015,
  gbnamefmt=lowercase,
  gbpub=false,
  doi=false,
  url=false,
  eprint=false,
  isbn=false,
]{biblatex}

% 参考文献引用文件位于 misc/ref.bib
\addbibresource{misc/ref.bib}

% 西文字体默认为 Times New Roman
\setromanfont{Times New Roman}

% 使用根目录的字体文件替代网络上的
\let\songti\relax
\let\heiti\relax
\let\kaishu\relax
\newCJKfontfamily\songti{SimSun.ttc}[AutoFakeSlant,AutoFakeBold]
\newCJKfontfamily\kaishu{SimKai.ttf}[AutoFakeSlant,AutoFakeBold]
\newCJKfontfamily\heiti{SimHei.ttf}[AutoFakeSlant,AutoFakeBold]

% 主题页面格式:NCHUThesis
\fancypagestyle{NCHUThesis}{
  % 设置空页眉
  \fancyhead{}
  % 删除页眉横线
  \renewcommand{\headrulewidth}{0pt}
  % 页码高度(不完美,比规定稍微靠下 2mm)
  \setlength{\footskip}{14pt}
  % 定义页码
  \fancyfoot[C]{\songti\zihao{-5} \thepage}
}

% 二级标题:楷体,小四号,加粗,汉字编号;间距:段前 0.5 行,段后 0 行;
\ctexset{section={
    number = {\Roman{section}},
    format = {\kaishu \raggedright \bfseries \zihao{4}},
    aftername = \heiti{、},
    beforeskip = 20bp plus 1ex minus .2ex,
    afterskip = 18bp plus .2ex,
    fixskip = true,
  }
}

% 设置目录样式
% 添加 PDF 链接
\addtocontents{toc}{\protect\hypersetup{hidelinks}}

% 修改超链接、引用的颜色
\hypersetup{
  colorlinks=true,
  linkcolor=black,
  anchorcolor=black,
  citecolor=black
}

% 正文页面
\renewcommand{\mainmatter}{
  \pagenumbering{arabic}
  \pagestyle{NCHUThesis}
}

% % 设置 caption 与 figure 之间的距离
% \setlength{\abovecaptionskip}{11pt}
% \setlength{\belowcaptionskip}{9pt}

% % 设置图片的 caption 格式
% \renewcommand{\thefigure}{\thechapter-\arabic{figure}}
% \captionsetup[figure]{font=small,labelsep=space}

% % 设置表格的 caption 格式和 caption 与 table 之间的垂直距离
% \renewcommand{\thetable}{\thechapter-\arabic{table}}
% \captionsetup[table]{font=small,labelsep=space,skip=2pt}

%%% ---- 图表标题设置 ----- %%%
\RequirePackage[labelsep=quad]{caption}     % 序号之后空一格写标题
% 设置表格标题字体为黑体, 设置图标题字体为宋体
\DeclareCaptionFont{heiti}{\heiti}
\captionsetup[table]{textfont=heiti}
\renewcommand\figurename{\songti\zihao{-4} 图}  
\renewcommand\tablename{\heiti\zihao{-4} 表} 

% 使用tabularx创建占满宽度的表格
\RequirePackage{tabularx, makecell}
\newcolumntype{L}{X}
\newcolumntype{C}{>{\centering \arraybackslash}X}
\newcolumntype{R}{>{\raggedleft \arraybackslash}X}

\RequirePackage{longtable}  % 做长表格的包
\RequirePackage{booktabs}   % 做三线表的包

% 列表样式
\RequirePackage{enumerate, enumitem}
\setlist{noitemsep}

% 调整底层 TeX 排版引擎参数以保证所有段落能够很好地以两端对齐的方式呈现
\tolerance=1
\emergencystretch=\maxdimen
\hyphenpenalty=10000
\hbadness=10000

% 设置数学公式编号格式
\renewcommand{\theequation}{\arabic{chapter}.\arabic{equation}}

\newcommand{\unnumchapter}[1]{
  \chapter*{\vskip 10bp\textmd{#1} \vskip -6bp}
  \addcontentsline{toc}{chapter}{#1}
  \stepcounter{chapter}
}

% 公式引用使用中文括号
\renewcommand{\eqref}[1]{\textup{{\normalfont(\ref{#1})\normalfont}}}

%%% ---- 引入宏包 ----- %%%
\RequirePackage{amsmath, amssymb}
\RequirePackage[amsmath,thmmarks]{ntheorem}  % 定理
\RequirePackage{graphicx, subcaption}
\RequirePackage{listings}                    % 代码段
% \RequirePackage{minted}                    % 代码高亮(需要 python 安装 pygments 库)
\RequirePackage[ruled,vlined]{algorithm2e}
\RequirePackage{algorithmic}    % 算法代码
\RequirePackage{tikz, pgfplots}              % 绘图
\RequirePackage{fontspec, color, url, array}

\RequirePackage{txfonts}                     % Times 风格(数学)字体

%%% ---- 定义字体 ----- %%%
\renewcommand{\normalsize}{\zihao{-4}}         % 正常字号
% 设置英文字体为 Times New Roman
\setmainfont[Ligatures=Rare]{Times New Roman}
\setsansfont[Ligatures=Rare]{Times New Roman}
\setmonofont[Ligatures=Rare]{Times New Roman}

% 算法两字用中文显示
\renewcommand{\algorithmcfname}{算法}

\lstdefinestyle{code}{
	backgroundcolor=\color{gray!10},
	commentstyle=\color{green!50!black},
	keywordstyle=\color{blue},
	stringstyle=\color{magenta},
	basicstyle=\linespread{1}\footnotesize\ttfamily,
	numberstyle=\tiny,
	breakatwhitespace=false,
	breaklines=true,
	captionpos=t,
	frame=single,
	keepspaces=true,
	language=java,
	numbers=none,
	numbersep=5pt,
	showspaces=false,
	showstringspaces=false,
	showtabs=false,
	tabsize=2,
	aboveskip=1em,
	belowskip=1em,
	belowcaptionskip=12pt
}

% 修改脚注
\makeatletter%
\long\def\@makefnmark{%
\hbox {{\normalfont \textsuperscript{\circled{\@thefnmark}}}}}%
\makeatother
\makeatletter%
\long\def\@makefntext#1{%
  \parindent 1em\noindent \hb@xt@ 1.8em{\hss \circled{\@thefnmark}}#1}%
\makeatother
\skip\footins=10mm plus 1mm
\footnotesep=6pt
\renewcommand{\footnotesize}{\songti\zihao{5}}
\renewcommand\footnoterule{\vspace*{-3pt}\hrule width 0.3\columnwidth height 1pt \vspace*{2.6pt}}

\newcommand*\circled[1]{\tikz[baseline=(char.base)]{%
\node[shape=circle,draw,inner sep=0.5pt] (char) {#1};}} % 圆圈数字①

%%% ---- 数学定理样式 ----- %%%
\theoremstyle{plain}
\theoremheaderfont{\heiti}
\theorembodyfont{\songti} \theoremindent0em
\theorempreskip{0pt}
\theorempostskip{0pt}
\theoremnumbering{arabic}
%\theoremsymbol{} %定理结束时自动添加的标志
\newtheorem{theorem}{\hspace{2em}定理}[section]
\newtheorem{definition}{\hspace{2em}定义}[section]
\newtheorem{lemma}{\hspace{2em}引理}[section]
\newtheorem{corollary}{\hspace{2em}推论}[section]
\newtheorem{proposition}{\hspace{2em}性质}[section]
\newtheorem{example}{\hspace{2em}例}[section]
\newtheorem{remark}{\hspace{2em}注}[section]

\theoremstyle{nonumberplain}
\theoremheaderfont{\heiti}
\theorembodyfont{\normalfont \rm \songti}
\theoremindent0em \theoremseparator{\hspace{1em}}
\theoremsymbol{$\square$}
\newtheorem{proof}{\hspace{2em}证明}

% 文档开始
\begin{document}
% 正文开始
\mainmatter
% 正文 22 磅的行距
\setlength{\parskip}{0em}
\renewcommand{\baselinestretch}{1.53}
% 修复脚注出现跨页的问题
\interfootnotelinepenalty=10000
%%
% NCHU Bachelor Proposal Report Template
%
% 南昌航空大学毕业设计任务书 (封面页) —— 使用 XeLaTeX 编译
%
% Copyright 2023 Arnold Chow
%
% The Current Maintainer of this work is Arnold Chow.
%
% 封面
%
% 如无特殊需要,本页面无需更改

% Underline new command for student information
% Usage: \dunderline[<offset>]{<line_thickness>}
\newcommand\dunderline[3][-1pt]{{%
  \setbox0=\hbox{#3}
  \ooalign{\copy0\cr\rule[\dimexpr#1-#2\relax]{\wd0}{#2}}}}


\makeatletter
\vspace*{-18mm}
\begin{center}
    \includegraphics[width=8cm]{images/header.png}
\vspace*{-6mm}

\zihao{1}\textbf{\songti{毕业设计(论文)任务书}}
\end{center}


\zihao{-4}\kaishu

% 按二级标题添加
% 题目:
\input{sections/1_title.tex}
% 使用的原始资料(数据)及设计技术要求:
\input{sections/2_origin_data.tex}
% 工作内容及进度安排
%%
% NCHU Bachelor Proposal Report Template
%
% 南昌航空大学毕业设计任务书 —— 使用 XeLaTeX 编译
%
% Copyright 2023 Arnold Chow
%
% The Current Maintainer of this work is Arnold Chow.
%
% Compile with: xelatex -> biber -> xelatex -> xelatex
\section{毕业设计(论文)工作内容及进度安排:}
xxx

xxx

% 主要参考资料
%%
% NCHU Bachelor Mission Statement Template
%
% 南昌航空大学毕业设计任务书 —— 使用 XeLaTeX 编译
%
% Copyright 2023 Arnold Chow
%
% The Current Maintainer of this work is Arnold Chow.
%
% Compile with: xelatex -> biber -> xelatex -> xelatex

\section{主要参考资料:}
\vspace*{-3ex}
\begin{center}
    \rule{16cm}{0.4pt}
\end{center}
\vspace*{-3ex}
\AllUnderline{16cm}{1.5mm}{
    \hspace{3.2ex} 这是一段测试。这是一段测试。这是一段测试。
    这是一段测试。这是一段测试。这是一段测试。这是一段测试。
    这是一段测试。这是一段测试。这是一段测试。这是一段测试。
    这是一段测试。这是一段测试。
    }

% 信息
%%
% NCHU Bachelor Mission Statement Template
%
% 南昌航空大学毕业设计任务书 —— 使用 XeLaTeX 编译
%
% Copyright 2023 Arnold Chow
%
% The Current Maintainer of this work is Arnold Chow.
%
% Compile with: xelatex -> biber -> xelatex -> xelatex


% 添加个人信息
\newcommand{\deptName}{测试与光电工程}
\newcommand{\majorName}{生物医学工程}
\newcommand{\class}{190841}
% 开始日期
\newcommand{\startYear}{2023}
\newcommand{\startMonth}{01}
\newcommand{\startDay}{01}
% 结束日期
\newcommand{\overYear}{2023}
\newcommand{\overMonth}{06}
\newcommand{\overDay}{30}

\vspace*{4ex}

\setlength{\parindent}{0pt} 

\selectfont{\dunderline[-5pt]{0.4pt}{\makebox[48mm][c]{\deptName}}}
\textbf{学院}
\selectfont{\dunderline[-5pt]{0.4pt}{\makebox[48mm][c]{\majorName}}}
\textbf{专业类}
\selectfont{\dunderline[-5pt]{0.4pt}{\makebox[32mm][c]{\class}}}
\textbf{班}

\vspace*{4ex}

\textbf{学生 (签名):}

\vspace*{4ex}

\textbf{日期:} \hspace{26mm} \textbf{自}
\selectfont{\dunderline[-5pt]{0.4pt}{\makebox[16mm][c]{\startYear}}}
\textbf{年}
\selectfont{\dunderline[-5pt]{0.4pt}{\makebox[8mm][c]{\startMonth}}}
\textbf{月}
\selectfont{\dunderline[-5pt]{0.4pt}{\makebox[8mm][c]{\startDay}}}
\textbf{日起至}
\selectfont{\dunderline[-5pt]{0.4pt}{\makebox[16mm][c]{\overYear}}}
\textbf{年}
\selectfont{\dunderline[-5pt]{0.4pt}{\makebox[8mm][c]{\overMonth}}}
\textbf{月}
\selectfont{\dunderline[-5pt]{0.4pt}{\makebox[8mm][c]{\overDay}}}
\textbf{日}

\vspace*{4ex}

\textbf{指导教师 (签名):}

\vspace*{4ex}

\textbf{助理指导教师 (并指出负责的部分):}

\vspace*{4ex}

\rightline{
\selectfont{\dunderline[-5pt]{0.4pt}{\makebox[48mm][c]{\majorName}}}
\textbf{系 (部) 主任 (签名):}
\selectfont{\dunderline[-5pt]{0.4pt}{\makebox[48mm][c]{ }}}
}

\vspace*{4ex}

\textbf{附注:任务书应该附在已完成的毕业设计说明书首页。}

\end{document}
